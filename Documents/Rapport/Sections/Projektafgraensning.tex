\section{Projektafgrænsning}
I forbindelse med dette projekt vil der blive udarbejdet noget\\
Eftersom vi ikke har DSP, Kom, og CPP længere kunne det være nice at få ændret projektafgrænsningen :D



\subsection{Arbejds- og fagområder}
Pan-Tilt projektet bliver udarbejdet for at give en bredere og bedre forståelse for det 4. semesters fag, derfor er der valgt nogle arbejds- og fag- områder som sørger for at projektet kommer til at have indhold fra hvert fag.

\subsubsection{Software}

\begin{itemize}[noitemsep]
  \item Digital signalbehandling
  \begin{itemize}[noitemsep]
    \item Filtrering af signalet for optimal transmission
    \item Problematikker i overgangsfasen mellem DTMF-toner
  \end{itemize}
  \item Datakommunikation
  \begin{itemize}[noitemsep]
    \item Protokoller
    \item Flow- og fejlkontrol
    \item Adressering i forbindelse med flere modtagere
    \item Kollision af datapakker skal undgås
  \end{itemize}
  \item C++
    \begin{itemize}[noitemsep]
    \item Mulighed for overvågning af transmissionen i forbindelse med debugging
    \item Objekt-orienteret programmering
    \item Dynamisk tilpasning af signal ift. afstand og vinkel
    \item Udvikling af brugerinterface til chat-klient
  \end{itemize}
\end{itemize}

\subsubsection{Hardware}

\begin{itemize}[noitemsep]
  \item Signalstyrkens påvirkning på datatransmission.
  \item SNR (tolerancen, målinger og beregninger)
  \begin{itemize}[noitemsep]
  \item Hvordan opnås en hurtig og effektiv transmission.
  \item Overvejelser omkring båndbredde.
  \end{itemize}
  \item Genklangs inflydelse på transmissionen
  \item Vinkling og afstand
  \begin{itemize}[noitemsep]
  \item Dynamisk og statisk placering af hardware
  \end{itemize}
  \item Ændring af transportmedie, eksempelvis ved forhindringer mellem sender og modtager.
\end{itemize}
