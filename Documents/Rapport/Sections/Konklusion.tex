\section{Konklusion}
Der er lavet et system, som følger objekter i rummet, så som den internationale rumstation, satelitter, solen og diverse planeter.\\
Dette er gjort muligt ved hjælp af en PID controller [ref variabler], som er blevet modelleret gennem matlab og simulink og grey box princippet, hvor variablene som skulle bruges i step respons test er blevet fundet. Ved hjælp af PID controlleren er det gjort muligt at have en præcision på $1/3$ grad, som er den højste præcision som de udleverede motorer er begrænset til. Denne controller er samtidig hurtig nok til at opfylde vores krav om hastighed, omend den ikke kunne opfylde vores krav til overshoot.

Det er samtidig lykkedes at opfylde de krav som blev stillet til FPGA'en, således at denne kan holde en adresse til diverse data som position, motor PWM, samt den kommunikerer med MPU'en via SPI.

Ved hjælp af koordinat oversættelse \ref{subsec:koordinat} er det muligt at tage de horizontale koordinater som er meget praktiske til at beskrive himmellegemer med, og omdanne dem til det koordinatsæt der giver bedst mening for selve systemet.