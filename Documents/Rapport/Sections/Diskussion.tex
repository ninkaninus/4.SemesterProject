\section{Diskussion}

Sammenlagt er projektet end med at blive meget lig hvad der blev sat ud på at skabe. Der er ikke mange ting der kunne være gjort meget anderledes, og som regel er det de mest optimale løsninger der er blevet valgt. Dette skal ikke være at sige at der ikke kan laves forbedringer, og som en næste itteration ville det mest optimale skridt nok være at lave om på pan og tilt systemet således at denne fik en større præcision og havde mindre slør, samt at udbygge brugervenligheden af systemet ved at bygge en GUI applikation der mere automatisk kunne tilbyde mulige mål samt give en mere overskuelig repræsentation af systemets tilstand ved brug af UART feedback.

Hvad der dog skulle gøres som afslutning af denne itteration ville være at udbygge systemet således at der benyttes et kamara der benytter fuld zoom, hvilket indebærer at der skal tages højde for at tilt systemet ikke kan bevæge sig frit rundt, så der skal sættes grændser så der er absolut sikkerhed for at kameraet ikke rammer rammen. Denne funktionalitet er inkorporeret, men skal blot aktiveres og justeres. Det indebærer dog at kontrol systemet skal rekalibreres, samt at der skal tilføjes en del vægt justering så systemet er balanceret.

I en senere itteration kunne kontrol systemet blive flyttet over i FPGA'en, såfremt det blev besluttet at dette var nødvendigt, men da det indtil videre har virket uden problemer ved at have den i microcontrolleren er dette hverken en nødvendig eller en tiltrækkende mulighed i denne del af processen.