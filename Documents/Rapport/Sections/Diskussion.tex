\section{Diskussion}

Forsøget på at udvikle et pan og tilt system der er i stand til at følge satelitter og himmellegemer er blevet opfyldt tilfredsstillende. Der er fundet gode koordinat systemer til både selve systemet og applikationen, og der er lavet en oversætter mellem disse. I forsøget på at skabe en controler med lav steady state fejl og overshoot er der blevet valgt en controller der indeholder både P, I og D. P delen er blevet brugt af stabilitets årsager, mens I delen benyttes til at fjerne steady state fejlen. D delen bruges for at reducerer overshoot, samt at skuppe vores poler ned på den reéle akse, og skuppe en pol langt med venstre, så denne har mindst mulig betydning. Dette betyder at vi nærmer os det 2. ordens system vi modellerer efter, og gør samtidig systemet mere stabilt.
\\
De mekaniske grændser i systemet er der taget højde for, da der både er en sikkerheds feature i FPGA'en der stopper enhver fejl der får systemet til at gå ud over sine grændser, og et koordinat oversættelses system der sikrer at systemet aldrig forsøger at gå ud over sine grændser. Der kunne aktiveres det redundante sikkerheds system der ligger i koordinat oversættelsen, men dette blev besluttet ikke var nødvendigt, da dette blot ville mindske det område der er tilgængeligt for pan og tilt systemet.
\\
I FPGA'en blev der stødt på det problem at der ikke eksisterer mulighed for at tristate interne signaler. Da hele det interne databus system er opbygget omkring netop dette, blev det nødvendigt at skifte over til at kører et format med to seperate databusser til at sende og modtage. Dette tilføjer en lille smule mere kompleksitet til den interne addressering, men fjerner en hel masse unødvendige fejlbeskeder i Xilinx. Og giver mulighed for at kunne simulerer vhdl koden korrekt.
\\
Den næste iteration ville indebærer at systemet blev modelleret med et fuld zoomet kamera, hvilket ville kræve en ny model af systemet og aktiveringen af sikkerheds featuren i koordinat systemet. Yderligere skal der så tilføjes en del vægt justering, da masse midtpunktet ville flytte sig meget ved et sådanne zoom, hvilket har enorm betydning på kontrol systemet. En GUI interface applikation på client computeren ville også være et ideélt forslag til næste iteration.




%Sammenlagt er projektet end med at blive meget lig hvad der blev sat ud på at skabe. Der er ikke mange ting der kunne være gjort meget anderledes, og som regel er det de mest optimale løsninger der er blevet valgt. Dette skal ikke være at sige at der ikke kan laves forbedringer, og som en næste itteration ville det mest optimale skridt nok være at lave om på pan og tilt systemet således at denne fik en større præcision og havde mindre slør, samt at udbygge brugervenligheden af systemet ved at bygge en GUI applikation der mere automatisk kunne tilbyde mulige mål samt give en mere overskuelig repræsentation af systemets tilstand ved brug af UART feedback.

%Hvad der dog skulle gøres som afslutning af denne itteration ville være at udbygge systemet således at der benyttes et kamara der benytter fuld zoom, hvilket indebærer at der skal tages højde for at tilt systemet ikke kan bevæge sig frit rundt, så der skal sættes grændser så der er absolut sikkerhed for at kameraet ikke rammer rammen. Denne funktionalitet er inkorporeret, men skal blot aktiveres og justeres. Det indebærer dog at kontrol systemet skal rekalibreres, samt at der skal tilføjes en del vægt justering så systemet er balanceret.

%I en senere itteration kunne kontrol systemet blive flyttet over i FPGA'en, såfremt det blev besluttet at dette var nødvendigt, men da det indtil videre har virket uden problemer ved at have den i microcontrolleren er dette hverken en nødvendig eller en tiltrækkende mulighed i denne del af processen.